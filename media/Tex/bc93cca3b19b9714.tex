\documentclass[preview]{standalone}

\usepackage[english]{babel}
\usepackage[utf8]{inputenc}
\usepackage[T1]{fontenc}
\usepackage{lmodern}
\usepackage{amsmath}
\usepackage{amssymb}
\usepackage{dsfont}
\usepackage{setspace}
\usepackage{tipa}
\usepackage{relsize}
\usepackage{textcomp}
\usepackage{mathrsfs}
\usepackage{calligra}
\usepackage{wasysym}
\usepackage{ragged2e}
\usepackage{physics}
\usepackage{xcolor}
\usepackage{microtype}
\DisableLigatures{encoding = *, family = * }
\linespread{1}

\begin{document}

\begin{center}
This is a way to specify a certain interval above a specified note, usually this note will be fundamental to the underlying tonal structure, such as the key's root tone. Thus given a key's root $ \overline{\mathcal{K}}$

	\[
	\boxed{x^{\hspace{-0.1cm}\urcorner} = \overline{\mathcal{K}} + x}
	\]

  For example, if $ \overline{\mathcal{K}} = \overline{9}$, then

	$3^{\hspace{-0.1cm}\urcorner} = \overline{9} + 3 = \overline{0}$
\end{center}

\end{document}
