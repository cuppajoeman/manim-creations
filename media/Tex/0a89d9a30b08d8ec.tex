\documentclass[preview]{standalone}

\usepackage[english]{babel}
\usepackage[utf8]{inputenc}
\usepackage[T1]{fontenc}
\usepackage{lmodern}
\usepackage{amsmath}
\usepackage{amssymb}
\usepackage{dsfont}
\usepackage{setspace}
\usepackage{tipa}
\usepackage{relsize}
\usepackage{textcomp}
\usepackage{mathrsfs}
\usepackage{calligra}
\usepackage{wasysym}
\usepackage{ragged2e}
\usepackage{physics}
\usepackage{xcolor}
\usepackage{microtype}
\DisableLigatures{encoding = *, family = * }
\linespread{1}

\begin{document}

\begin{center}
This notation doesn't really convey what's going on. In this situation $ \overline{9} \mid 0 \ 4 \ 7 \ 11$ is going to have the same effect as $ \overline{0} \mid 0 \ 4 \ 7 \ 11$ as they're both a chord which is constructed of the same intervals from the root of the key.
\end{center}

\end{document}
