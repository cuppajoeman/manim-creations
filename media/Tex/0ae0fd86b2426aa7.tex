\documentclass[preview]{standalone}

\usepackage[english]{babel}
\usepackage[utf8]{inputenc}
\usepackage[T1]{fontenc}
\usepackage{lmodern}
\usepackage{amsmath}
\usepackage{amssymb}
\usepackage{dsfont}
\usepackage{setspace}
\usepackage{tipa}
\usepackage{relsize}
\usepackage{textcomp}
\usepackage{mathrsfs}
\usepackage{calligra}
\usepackage{wasysym}
\usepackage{ragged2e}
\usepackage{physics}
\usepackage{xcolor}
\usepackage{microtype}
\DisableLigatures{encoding = *, family = * }
\linespread{1}

\begin{document}

\begin{center}
So let's suppose our anchor point is on the $\overline{4_L}$ string, then moving over 1 string increases us by 5 semitones, if we move over another string we will move up by another 5 semitones, that is in relation to the $\overline{4_L}$ string, this new note is 10 semitones higer, to get to the $\overline{7}$ string from the $\overline{4_L}$ string you will add $(5 + 5 + 5) = 3$ semitones, by filling out this whole fret position, we get ...
\end{center}

\end{document}
